%%%%%%%%%%%%%%%%%%%%%%%%%%%%%%%%%%%%%%%%%
% Memo
% LaTeX Template
% Version 1.0 (30/12/13)
%
% This template has been downloaded from:
% http://www.LaTeXTemplates.com
%
% Original author:
% Rob Oakes (http://www.oak-tree.us) with modifications by:
% Vel (vel@latextemplates.com)
%
% License:
% CC BY-NC-SA 3.0 (http://creativecommons.org/licenses/by-nc-sa/3.0/)
%
%%%%%%%%%%%%%%%%%%%%%%%%%%%%%%%%%%%%%%%%%

\documentclass[A4,12pt]{texMemo} % Set the paper size (letterpaper, a4paper, etc) and font size (10pt, 11pt or 12pt)

\usepackage{parskip} % Adds spacing between paragraphs
\setlength{\parindent}{15pt} % Indent paragraphs

%----------------------------------------------------------------------------------------
%	MEMO INFORMATION
%----------------------------------------------------------------------------------------

\memoto{ILS 2021} % Recipient(s)

\memofrom{Kelompok Pantai Nanyi, Puri Anom} % Sender(s)

\memosubject{Pantai Pasir Hitam, Pantai Nanyi} % Memo subject

% \memodate{Monday, December 30, 2013} % Date, set to \today for automatically printing todays date

% \logo{\includegraphics[width=0.3\textwidth]{logo.png}} % Institution logo at the top right of the memo, comment out this line for no logo

%----------------------------------------------------------------------------------------

\begin{document}

\maketitle % Print the memo header information

%----------------------------------------------------------------------------------------
%	MEMO CONTENT
%----------------------------------------------------------------------------------------

Pantai Nanyi... ya... Mendengar nama Pantai Nanyi, mungkin masih asing ya di telinga sebagian orang. Nama Pantai Nanyi memang belum sepopuler jika dibandingkan dengan beberapa pantai di Bali seperti Pantai Balangan, Pantai Kuta, Pantai Pandawa. Mungkin agak sedikit lucu kalau anda mendengar nama pantai Nyanyi seperti nyanyian. Namun sesungguhnya nama Pantai Nyanyi ini menyuguhkan nyanyian keindahan Pantai yang tersembunyi dan patut di eksplorasi. Asal usul nama pantai Nyanyi diambil dari sebuah nama desa yang tempatnya berada di dusun nyanyi, Desa Beraban, Kecamatan Kediri, Tabanan Bali. 

Salah satu yang menjadi daya tarik wisata mengapa banyak orang menyukai untuk berwisata ke Bali karena daya tariknya dan pesona kecantikan pantainya yang eksotis berwarna hitam yang legam dan halus. Pantai Nyanyi memiliki keunikan tersendiri dan berbeda dengan pantai yang lainnya dimana pantai nyanyi menghadap ke arah selatan lautan biru Samudera Hindia. 

Pantai yang berhadapan langsung dengan Samudera Hindia yang terkenal dengan gelombang ombak yang besar menjadikan Pantai Nyanyi sangat ideal dan pas bagi tempat surfing di Bali. Oleh karenanya jangan heran jika suatu hari anda berkunjung ke ini anda akan mendapati beberapa wisatawan yang khususnya turis asing yang bermain selancar air disini.

Adakalanya mungkin anda bosan untuk berkunjung ke pantai pantai yang ramai dan mungkin ingin menikmati kesunyian dan ketenangan dan jauh dari hiruk pikuk ramainya kota. Pantai Nyanyi memang jauh dari dari pusat keramaian serta rumah-rumah nelayan dan warga penduduk sekitarnya. Nuansa pantai yang menawarkan alam yang tenang dan damai serta privasi sangat ideal bagi mereka yang membutuhkan tempat relaksasi dan refreshing. 

Setiap pengunjung yang masuk ke area Pantai Nanyi tidak dikenakan biaya alias gratis. Ada banyak aktivitas seru yang bisa anda lakukan di Pantai Nanyi Tabnanan Bali. Seperti berenang berjemur, surfing, serta jalan-jalan menyusuri pantai sambil menikmati sunset. Saking nyaman dan damainya, Pantai Nanyi ini sering dijadikan tempat wisata anak dan tempat wisata keluarga bagi turis domestik dan turis asing. 

%----------------------------------------------------------------------------------------

\end{document}
